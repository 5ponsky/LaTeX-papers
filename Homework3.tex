% !TEX TS-program = pdflatex
% !TEX encoding = UTF-8 Unicode

% This is a simple template for a LaTeX document using the "article" class.
% See "book", "report", "letter" for other types of document.

\documentclass[20pt]{article} % use larger type; default would be 10pt

\usepackage[utf8]{inputenc} % set input encoding (not needed with XeLaTeX)

%%% Examples of Article customizations
% These packages are optional, depending whether you want the features they provide.
% See the LaTeX Companion or other references for full information.

%%% PAGE DIMENSIONS
\usepackage{geometry} % to change the page dimensions
\geometry{a4paper} % or letterpaper (US) or a5paper or....
% \geometry{margin=2in} % for example, change the margins to 2 inches all round
% \geometry{landscape} % set up the page for landscape
%   read geometry.pdf for detailed page layout information

\usepackage{graphicx} % support the \includegraphics command and options

% \usepackage[parfill]{parskip} % Activate to begin paragraphs with an empty line rather than an indent

%%% PACKAGES
\usepackage{booktabs} % for much better looking tables
\usepackage{array} % for better arrays (eg matrices) in maths
\usepackage{paralist} % very flexible & customisable lists (eg. enumerate/itemize, etc.)
\usepackage{verbatim} % adds environment for commenting out blocks of text & for better verbatim
\usepackage{subfig} % make it possible to include more than one captioned figure/table in a single float
\usepackage{mathtools}
% These packages are all incorporated in the memoir class to one degree or another...

%%% Other stuff
\DeclarePairedDelimiter\ceil{\lceil}{\rceil}
\DeclarePairedDelimiter\floor{\lfloor}{\rfloor}

%%% HEADERS & FOOTERS
\usepackage{fancyhdr} % This should be set AFTER setting up the page geometry
\pagestyle{fancy} % options: empty , plain , fancy
\renewcommand{\headrulewidth}{0pt} % customise the layout...
\lhead{}\chead{}\rhead{}
\lfoot{}\cfoot{\thepage}\rfoot{}

%%% SECTION TITLE APPEARANCE
\usepackage{sectsty}
\allsectionsfont{\sffamily\mdseries\upshape} % (See the fntguide.pdf for font help)
% (This matches ConTeXt defaults)

%%% ToC (table of contents) APPEARANCE
\usepackage[nottoc,notlof,notlot]{tocbibind} % Put the bibliography in the ToC
\usepackage[titles,subfigure]{tocloft} % Alter the style of the Table of Contents
\renewcommand{\cftsecfont}{\rmfamily\mdseries\upshape}
\renewcommand{\cftsecpagefont}{\rmfamily\mdseries\upshape} % No bold!

%%% END Article customizations

%%% nice things to keep around

% \noindent\rule{2cm}{0.4pt} %%% puts a small horizontal line
% \mathcal{O} %%% big O notation

%%% The "real" document content comes below...

\title{Algorithms Homework 2}
\author{Liam Dillingham}
%\date{} % Activate to display a given date or no date (if empty),
         % otherwise the current date is printed 

\begin{document}
\maketitle

\section{Question 8.2-4}
Describe an algorithm that, given $n$ integers in the range $0$ to $k$, preprocesses its input and then answers any query about how many of the $n$ integers fall into a range $[a..b]$ in $\mathcal{O}(1)$ time. Your algorithm should use $\Theta(n+k)$ preprocessing time \\ 
\noindent\rule{2cm}{0.4pt} \\ 

Suppose we have a set of integers, $S$, where the $min(S) = 0$, and $max(S) = k$.  We want an algorithm that can tell how how many integers, $m$ fall in the range $[a,b]$ such that 
$0 \leq a \leq b \leq k$ in time $\mathcal{O}(1)$. \\

Let us observe COUNTING-SORT.  First, we initialize an array of size $k$ such that every element is equal to $0$. Then we populate the array with the frequency in which each element appears.  Then, for any element $j \in [0,k]$, we have another loop to sum the number of elements less than or equal to $j$ with the frequency of $j$.  \\

Note that since an array is already ordered by its own index.  Since this is true, then we can count the number of elements $m$ that fall within the range $[a,b]$ by simply iterating across the "counting" array: $m$ += $(C[b]$ downto $C[a])$. Note that the difference between two constants is also a constant. \\

The counting step (building the frequency table) takes $\Theta(n)$ because we must loop over the entire input array, and the frequency step takes $\Theta(k)$ because we must loop over the size of the counting array.  Thus the preprocessing step will take $\Theta(n+k)$, and the query step will take $\Theta(b-a)$, or $\mathcal{O}(1)$.

\newpage
\section{Question 8.3-4}
Show how to sort $n$ integers in the range $0$ to $n^{3} -1$ in $\mathcal{O}(n)$ time. \\ 
\noindent\rule{2cm}{0.4pt} \\ 

Note: According to lemma 8.3: "Given $n$ d-digit numbers in which each digit can take on up to k possible values, RADIX-SORT correctly sorts these numbers in $\Theta(d(n+k))$ time if the stable sort it uses takes $\Theta(n+k)$ time" \\

if we distribute the $d$, we get $\Theta(dn+dk)$. Note that if $n$ is sufficiently large, then the number of digits in each integer could be at least $n$


\section{Question 8.4-2}
Explain why the worst-case running time for bucket sort is $\Theta(n^{2})$. What simple change to the algorithm preserves its linear average-case running time and makes its worst-case running time $\mathcal{O}(nlgn)$? \\ 
\noindent\rule{2cm}{0.4pt} \\ \\



\end{document}
