\documentclass[oneside,11pt]{amsart}

\usepackage{url}

\usepackage{amsmath}
\usepackage{amssymb}
\usepackage{amsthm}
\usepackage{graphics,graphicx}
\usepackage{verbatim}

\topmargin=0in
\oddsidemargin=0in
\evensidemargin=0in
\textwidth=6.5in 
\textheight=8.5in


\newcommand{\FF}{\mathbf F}
\newcommand{\ZZ}{\mathbf Z}
\newcommand{\RR}{\mathbf R}
\newcommand{\QQ}{\mathbf Q}
\newcommand{\CC}{\mathbf C}
\newcommand{\NN}{\mathbf N}
\newcommand{\Hom}{\operatorname{Hom}}
\newcommand{\lcm}{\operatorname{lcm}}

\newcommand{\GL}{\operatorname{GL}}
\newcommand{\SL}{\operatorname{SL}}
\newcommand{\Heis}{\operatorname{Heis}}


\newcommand{\im}{\operatorname{im}}
\newcommand{\Aut}{\operatorname{Aut}}


\DeclareFontFamily{OT1}{rsfs}{}
\DeclareFontShape{OT1}{rsfs}{n}{it}{<-> rsfs10}{}
\DeclareMathAlphabet{\mathscr}{OT1}{rsfs}{n}{it}



%Customized Theorem Environments
\newtheoremstyle%
{custom}%
{}%                         Space above
{}%													Space below
{}%													Body font
{}%                         Indent amount
{}%                         Theorem head font
{.}%                        Punctuation after heading
{ }%                        Space after heading
{\thmname{}%                Additional specifications for theorem head
\thmnumber{}%
\thmnote{\bfseries #3}}%

\newtheoremstyle%
{Theorem}%
{}%
{}%
{\itshape}%
{}%
{}%
{.}%
{ }%
{\thmname{\bfseries #1}%
\thmnumber{\;\bfseries #2}%
\thmnote{\;(\bfseries #3)}}%

%Theorem Environments
\theoremstyle{Theorem}
\newtheorem{thm}{Theorem}[section]
\newtheorem{cor}[thm]{Corollary}
\newtheorem{lem}[thm]{Lemma}
%\newtheorem{prop}{Proposition}[section]
\theoremstyle{definition}
\newtheorem{dff}[thm]{Definition}
\newtheorem{xmp}[thm]{Example}
\newtheorem{remark}[thm]{Remark}
\newtheorem{conj}[thm]{Conjecture}


\def\mbf{\mathbf}
\def\mcal{\mathcal}

\def\fa{\mathfrak{a}}
\def\fb{\mathfrak{b}}
\def\fn{\mathfrak{n}}
\def\fp{\mathfrak{p}}

\def\nsg{\vartriangleleft}
\def\tn{\textnormal}

\newcommand{\Sym}{\operatorname {Sym}}
\newcommand{\Tr}{\operatorname {Tr}}

\newcommand{\tmat}[4]{ \left( \begin{smallmatrix} #1 & #2 \\ #3 & #4 \end{smallmatrix} \right)}

\def\cB{\mathcal{B}}
\def\cC{\mathcal{C}}
\def\cD{\mathcal{D}}


\title{Template: Math 3113, Homework 2}
\author{ Author = You}

\begin{document}

\maketitle

\vspace{-5mm}




\begin{enumerate}

\item The following questions concern finding formula and utilizing induction. 
\begin{enumerate}
\item For $g$ and $h$ in a group $G$, prove that $(gh)^n = g (hg)^{n-1} h$ for all $n > 0$. 
\begin{proof}

\end{proof}
\item For $g$ and $h$ in a group $G$, prove that $(gh)^n = e$ if and only if $(hg)^n = e$. Does this force $gh = hg$? 
\begin{proof}

\end{proof}

\end{enumerate}

\newpage


\item What is the cycle decomposition of $$ \left( \begin{matrix} 1 & 2 & 3 & 4 & 5 & 6 & 7 & 8 \\ 5 & 7 & 8 & 3 & 1 & 2 & 6 & 4 \end{matrix} \right) .$$

\begin{proof}

\end{proof}


\newpage


\item  Suppose $G$ is a cyclic group of order $144$. How many elements are in the subgroup $\langle g^{26} \rangle$? How many subgroups contain this subgroup? 

\begin{proof}

\end{proof}

\newpage

\item Which of the following subsets are subgroups of $\GL_2(\RR)$?
\begin{enumerate}
\item $\left \{ \left( \begin{smallmatrix} a & b \\ 0 & c \end{smallmatrix} \right) \colon a, c \neq 0 \right \}.$

\item $\left \{ \left( \begin{smallmatrix} a & b \\ 1/b & 1/a \end{smallmatrix} \right) \colon a,b \neq 0 \right \}.$
\item $\left \{ \left( \begin{smallmatrix} a & b \\ c & d \end{smallmatrix} \right) \colon ad-bc = 1  \right \}.$
\item $\left \{ \left( \begin{smallmatrix} a & b \\ 0 & 1 \end{smallmatrix} \right) \colon a \neq 0  \right \}.$
\end{enumerate}
\begin{proof}

\end{proof}

\newpage


\item Fix $G = (\ZZ/7\ZZ)^\times$ and $g = a \bmod 7$ with $(a,7) = 1$. Consider the function $\varphi_a \colon G \to G$ defined by $\varphi( x \bmod 7) \mapsto 2x \bmod 7$. 
\begin{enumerate}
\item Show that $\varphi_a$ is a bijection. 
\begin{proof}

\end{proof}
\item Prove that $\varphi_a \circ \varphi_b = \varphi_{ab}$. 
\begin{proof}

\end{proof}
\item Show that the ordered set $\{ \varphi_a(1 \bmod 7), \ldots, \varphi_a(6 \bmod 7) \}$ is a permutation of the standard representatives modulo $7$. 
\begin{proof}

\end{proof}
\item What is the cycle decomposition of the permutation corresponding to $\varphi_3$? 
\begin{proof}

\end{proof}
\item Is every permutation in $S_6$ realized this way? 
\begin{proof}

\end{proof}
\end{enumerate}

\end{enumerate}



\end{document}
